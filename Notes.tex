\section{Notes}
\subsection{Hằng đẳng thức}

\begin{itemize}
	\item Hằng đẳng thức đáng nhớ:
	\begin{itemize}
		\item[$\circ$] $(a + b)^2 = a^2 + 2ab + b^2$.
		\item[$\circ$] $(a - b)^2 = a^2 - 2ab + b^2$.
		\item[$\circ$] $(a + b)^3 = a^3 + 3a^2b + 3ab^2 + b^3$.
		\item[$\circ$] $(a - b)^3 = a^3 - 3a^2b + 2ab^2 - b^3$.
		\item[$\circ$] $a^2 - b^2 = (a - b)(a + b)$.
		\item[$\circ$] $a^3 + b^3 = (a + b)(a^2 - ab + b^2)$.
		\item[$\circ$] $a^3 - b^3 = (a - b)(a^2 + ab + b^2)$.
	\end{itemize}
	\item Một số hằng đẳng thức khác:
	\begin{itemize}
		\item[$\circ$] $(a + b + c)^2 = a^2 + b^2 + c^2 + 2ab + 2bc + 2ca$.
		\item[$\circ$] $(a + b - c)^2 = a^2 + b^2 + c^2 + 2ab - 2bc - 2ca$.
		\item[$\circ$] $(a + b + c)^3 = a^3 + b^3 + c^3 + 3(a + b)(b + c)(c + a)$.
		\item[$\circ$] $\displaystyle a^2 + b^2 + c^2 - ab - bc - ca = \frac{(a - b)^2 + (b - c)^2 + (c - a)^2}{2}$.
		\item[$\circ$] $(a + b + c)(ab + bc + ca) = (a + b)(b + c)(c + a) + abc$.
		\item[$\circ$] $a^3 + b^3 + c^3 - 3abc = (a + b + c)(a^2 + b^2 + c^2 - ab - bc - ca)$.
		\item[$\circ$] $a^k - b^k = (a - b)(a^{k - 1} + a^{k - 2}b + ... + b^{k - 1})$
	\end{itemize}
\end{itemize}


\pagebreak
% ============================================= %
\subsection{Bất đẳng thức}
\subsubsection*{Một số bất đẳng thức thường dùng}

\begin{mybox}
\textbf{Bất đẳng thức AM - GM}: Với $a_1, a_2, ..., a_n \ge 0$ $(n \ge 2)$, ta có:
$$
	a_1 + a_2 + ... + a_n \ge n\sqrt[n]{a_1 a_2 ... a_n}
$$
Đẳng thức xảy ra khi và chỉ khi $a_1 = a_2 = ... = a_n$.

Dạng tương đương của \textbf{bất đẳng thức AM - GM}:
$$
	a_1 a_2 ... a_n \le \left(\frac{a_1 + a_2 + ... + a_n}{n}\right)^n
$$
\end{mybox}

\begin{mybox}
\textbf{Bất đẳng thức Cauchy - Schwarz}: Với $2n$ số thực $a_1, a_2, ... a_n$ và
$b_1, b_2, ... b_n$ tùy ý, ta có
$$
	(a_1 b_1 + a_2 b_2 + ... + a_n b_n)^2 \le (a_1^2 + a_2^2 + ... + a_n^2)(b_1^2 + b_2^2 + ... + b_n^2).
$$

Đẳng thức xảy ra khi và chỉ khi $\displaystyle \frac{a_1}{b_1} = \frac{a_2}{b_2} = ... = \frac{a_n}{b_n}$.

\item 
Một dạng khác thường dùng của \textbf{bất đẳng thức Cauchy - Schwarz}: Với $n$ số thực $a_1, a_2, ... a_n$ và $n$ số thực dương $b_1, b_2, ... b_n$, ta có
$$
	\sqrt{a_1 b_1} + \sqrt{a_2 b_2} + ... + \sqrt{a_n b_n} \le \sqrt{(a_1 + a_2 + ... + a_n)(b_1 + b_2 + ... b_n)}.
$$

Dạng cộng mẫu của \textbf{bất đẳng thức Cauchy - Schwarz}:  Với $n$ số thực $a_1, a_2, ... a_n$ và $n$ số thực dương $b_1, b_2, ... b_n$ ta có
$$
	\frac{a_1^2}{b_1} + \frac{a_2^2}{b_2} + ... + \frac{a_n^2}{b_n} \ge \frac{(a_1 + a_2 + ... + a_n)^2}{b_1 + b_2 + ... + b_n}.
$$
\end{mybox}

\begin{itemize}
	\item $a^2 + b^2 \ge 2ab$

	\item $a^3 + b^3 + c^3 \ge 3abc$

	\item $2(a^2 + b^2) \ge (a + b)^2 \ge 4ab$

	\item $a^2 + b^2 + c^2 \ge ab + bc + ca$

	\item $3(a^2 + b^2 + c^2) \ge (a + b + c)^2 \ge 3(ab + bc + ca)$

	\item $(ab + bc + ca)^2 \ge 3abc(a + b + c)$

	\item $a^2b^2 + b^2c^2 + c^2a^2 \ge abc(a + b + c)$

	\item $\displaystyle \frac{1}{a_1} + \frac{1}{a_2} + ... + \frac{1}{a_n} \ge \frac{n^2}{a_1 + a_2 + ... + a_n}$

	\item $\displaystyle (a_1 + a_2 + ... + a_n)\left(\frac{1}{a_1} + \frac{1}{a_2} + ... + \frac{1}{a_n} \right) \ge n^2$

	\item $\displaystyle (a + b)(b + c)(c + a) \ge \frac{8}{9}(a + b + c)(ab + bc + ca)$
\end{itemize}

% ============================================= %
\subsection{Các bài toán về đa thức}
\begin{itemize}
	\item
	Đa thức $f(x) = a_nx^n + a_{n - 1}x^{n - 1} + \ldots + a_1x + a_0$ có $a_n$ là hệ số cao nhất, $a_0$ là hệ số tự do và có bậc $n$.
	\item
	Số $c$ là nghiệm của đa thức nếu $f(c) = 0$.
	\item
	Nếu $a_i \in \Z \ \forall i$ thì ta gọi đa thức $f \in \Z[x]$ tức tập các đa thức hệ số nguyên.
	\item
	Với mọi đa thức $f(x)$ và mọi đa thức $g(x)$ khác $0$ luôn tìm được một đa thức $q(x)$ và đa thức $r(x)$ sao cho $f(x) = g(x).q(x) + r(x)$ $(deg(r) < deg(v))$.
\end{itemize}

\subsubsection{Một số tính chất cần nắm}
Với $f \in \Z[x]$ và $a, b$ là hai số nguyên khác nhau, ta luôn có $f(a) - f(b)$ chia hết cho $f(a - b)$.

\begin{mybox}
\begin{theorem}[\textbf{Bezout}]
	Phần dư trong phép chia đa thức $f(x) = a_nx^n + a_{n - 1}x^{n - 1} + \ldots + a_1x + a_0$ cho đa thức bậc nhất $(x - a)$ là một hằng số và bằng giá trị của đa thức $f(x)$ tại $x = a$.
\end{theorem}
\end{mybox}

\begin{corollary}[1]
Mọi đa thức bậc $n$ đều không có quá $n$ nghiệm thực.
\end{corollary}

\begin{corollary}[1.1]
Nếu đa thức $f(x)$ có bậc không quá $n$ mà có nhiều hơn $n$ nghiệm thì từng hệ số của đa thức $f(x)$ bằng $0$.
\end{corollary}

\begin{corollary}[1.2]
Nếu đa thức bậc $n$ nhận giá trị bằng nhau tại $(n + 1)$ giá trị của $x$ thì đa thức đó là đa thức hằng.
\end{corollary}

\begin{corollary}[2]
	Nếu $a$ là nghiệm của đa thức $f(x)$ khi đó ta có phép chia hết $f(x) = (x - a).g(x)$ trong đó $g(x) = b_0x^{n - 1} + b_1x^{n - 2} + b_{n - 2}x + b_{n - 1}$.
\end{corollary}

\begin{mybox}
\begin{theorem}[Viète]
	Giả sử đa thức $f(x) = a_nx^n + a_{n - 1}x^{n - 1} + \ldots + a_1x + a_0$ có các nghiệm $x_1, x_2, \ldots, x_n$. Khi đó ta có các đẳng thức sau:
	\begin{align*}
		x_1 + x_2 + \ldots + x_n = -\frac{a_{n - 1}}{a_n} \\
		x_1 x_2 + x_2 x_3 + \ldots + x_{n - 1} x_{n} = \frac{a_{n - 2}}{a_n} \\
		x_1 x_2 x_3 + x_2 x_3 x_4 + \ldots + x_{n - 2} x_{n - 1} x_n = -\frac{a_{n - 3}}{a_n} \\ 
		x_1 x_2 x_3 \ldots x_n = (-1)^n \frac{a_0}{a_n}
	\end{align*}
\end{theorem}

\begin{theorem}[Viète đảo]
	Nếu như các số thực $x_1, x_2, \ldots, x_n$ thỏa mãn hệ:
	\[
		S_k = (-1)^k \frac{a_{n - k}}{a_n},\ k = \overline{1, k} 
	\]
	Khi đó $x_1, x_2, \ldots, x_n$ là $n$ nghiệm của đa thức bậc $n$: $f(x) = a_nx^n + a_{n - 1}x^{n - 1} + \ldots + a_1x + a_0$.
\end{theorem}
\end{mybox}
